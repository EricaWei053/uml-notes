\usepackage{lmodern}
\usepackage{amssymb,amsmath,amsthm}
\usepackage{mathrsfs, bbm, tikz-cd}
\usepackage{ifxetex,ifluatex}
\usepackage{mdframed}
\usepackage{xcolor}
\usepackage{float}
\usepackage{multicol}
\usepackage{caption}
\usepackage{algorithm,algorithmic}
\usepackage{cancel} 
\usepackage{stackengine}



\let\oldemptyset\emptyset
\renewcommand{\emptyset}{\varnothing}

% set default figure placement to htbp
\makeatletter
\def\fps@figure{htbp}
\makeatother


\newmdenv[
  backgroundcolor=gray!10,
  topline=false,
  bottomline=false,
  rightline=false,
  leftline=false,
  innertopmargin=\topskip,
  innerbottommargin=\topskip
]{problem}

\newtheorem{theorem}{Theorem}[subsection]
\newtheorem{exercise}[theorem]{Exercise}
\newtheorem{conj}[theorem]{Conjecture}
\newtheorem{lemma}[theorem]{Lemma}
\newtheorem{corollary}[theorem]{Corollary}
\newtheorem{claim}[theorem]{Claim}
\newtheorem{proposition}[theorem]{Proposition}
\newtheorem{definition}[theorem]{Definition}
\newtheorem{remark}[theorem]{Remark}
\newtheorem{fact}[theorem]{Fact}
\newtheorem{example}[theorem]{Example}
\newtheorem{alg}[theorem]{Algorithm}
\newtheorem{result}[theorem]{Result}
\newtheorem{conjecture}[theorem]{Conjecture}
\newtheorem{notation}[theorem]{Notation}



\newcommand{\eps}{\varepsilon}
\newcommand{\xii}{x^{(i)}}
\newcommand{\yii}{y^{(i)}}
\newcommand{\cA}{\mathcal{A}}
\newcommand{\cB}{\mathcal{B}}
\newcommand{\cC}{\mathcal{C}}
\newcommand{\cE}{\mathcal{E}}
\newcommand{\cI}{\mathcal{I}}
\newcommand{\cP}{\mathcal{P}}
\newcommand{\cG}{\mathcal{G}}
\newcommand{\cM}{\mathcal{M}}
\newcommand{\cN}{\mathcal{N}}
\newcommand{\cD}{\mathcal{D}}
\newcommand{\cX}{\mathcal{X}}
\newcommand{\cS}{\mathcal{S}}
\newcommand{\cQ}{\mathcal{Q}}
\newcommand{\cU}{\mathcal{U}}
\newcommand{\cT}{\mathcal{T}}
\newcommand{\cR}{\mathcal{R}}
\newcommand{\cF}{\mathcal{F}}
\newcommand{\cV}{\mathcal{V}}
\newcommand{\cW}{\mathcal{W}}
\newcommand{\cZ}{\mathcal{Z}}
\newcommand{\ip}[1]{\left\langle #1 \right\rangle}
\newcommand{\SSS}{\mathbb{S}}
\newcommand{\abs}[1]{\left| #1 \right|}
\newcommand{\maxtwo}{\textstyle \max^{(2)}}

\newcommand{\F}{\mathbb{F}}
\newcommand{\R}{\mathbb{R}}
\newcommand{\Pat}{P\v{a}tra\c{s}cu }

\newcommand{\Var}{\mathrm{Var}}
\newcommand\E{\mathbb{E}}
\newcommand{\bbF}{\mathbb{F}}
\newcommand{\bbR}{\mathbb{R}}
\newcommand{\bbQ}{\mathbb{Q}}
\newcommand{\bbN}{\mathbb{N}}
\newcommand{\bbC}{\mathbb{C}}
\newcommand{\bbZ}{\mathbb{Z}}

\providecommand{\norm}[1]{\ensuremath{\left\lVert#1\right\rVert}}
\DeclareMathOperator*{\argmin}{\mathrm{arg\,min}}
\DeclareMathOperator*{\argmax}{\mathrm{arg\,max}}
\providecommand{\ceil}[1]{\ensuremath{\left\lceil #1 \right\rceil}}
% KL DIVERGENCE : 
\providecommand{\Div}[2]{\ensuremath{\mathsf{D}\left(\begin{array}{c} #1 \\ \hline  \hline  #2\end{array} \right)}}
\renewcommand{\div}[2]{\mathsf{D}\left( #1 || #2\right)} 

%\renewcommand{\cite}[1]{[#1]}
%\def\beginrefs{\begin{list}%
%        {[\arabic{equation}]}{\usecounter{equation}
%         \setlength{\leftmargin}{2.0truecm}\setlength{\labelsep}{0.4truecm}%
%         \setlength{\labelwidth}{1.6truecm}}}
%\def\endrefs{\end{list}}
%\def\bibentry#1{\item[\hbox{[#1]}]}


% code from:
% https://tex.stackexchange.com/questions/248412/how-to-write-matrices-with-dimensions-in-latex
\stackMath
\newlength\matfield
\newlength\tmplength
\def\matscale{1.}
\newcommand\dimbox[3]{%
  \setlength\matfield{\matscale\baselineskip}%
  \setbox0=\hbox{\vphantom{X}\smash{#3}}%
  \setlength{\tmplength}{#1\matfield-\ht0-\dp0}%
  \fboxrule=1pt\fboxsep=-\fboxrule\relax%
  \fbox{\makebox[#2\matfield]{\addstackgap[.5\tmplength]{\box0}}}%
}
\newcommand\raiserows[2]{%
   \setlength\matfield{\matscale\baselineskip}%
   \raisebox{#1\matfield}{#2}%
}
\newcommand\matbox[5]{
  \stackunder{\dimbox{#1}{#2}{$#5$}}{\scriptstyle(#3\times #4)}%
}


%%%   % code from:
%%%   %https://tex.stackexchange.com/questions/285578/how-to-draw-parallelepiped-and-cube-with-latex
%%%   \tikzset{
%%%     annotated cuboid/.pic={
%%%       \tikzset{%
%%%         every edge quotes/.append style={midway, auto},
%%%         /cuboid/.cd,
%%%         #1
%%%       }
%%%       \draw [every edge/.append style={pic actions, densely dashed, opacity=.5}, pic actions]
%%%       (0,0,0) coordinate (o) -- ++(-\cubescale*\cubex,0,0) coordinate (a) -- ++(0,-\cubescale*\cubey,0) coordinate (b) edge coordinate [pos=1] (g) ++(0,0,-\cubescale*\cubez)  -- ++(\cubescale*\cubex,0,0) coordinate (c) -- cycle
%%%       (o) -- ++(0,0,-\cubescale*\cubez) coordinate (d) -- ++(0,-\cubescale*\cubey,0) coordinate (e) edge (g) -- (c) -- cycle
%%%       (o) -- (a) -- ++(0,0,-\cubescale*\cubez) coordinate (f) edge (g) -- (d) -- cycle;
%%%       \path [every edge/.append style={pic actions, |-|}]
%%%       (b) +(0,-5pt) coordinate (b1) edge ["\cubelabely"'] (b1 -| c)
%%%       (b) +(-5pt,0) coordinate (b2) edge ["\cubelabelx"] (b2 |- a)
%%%       (c) +(3.5pt,-3.5pt) coordinate (c2) edge ["\cubelabelz"'] ([xshift=3.5pt,yshift=-3.5pt]e)
%%%       ;
%%%     },
%%%     /cuboid/.search also={/tikz},
%%%     /cuboid/.cd,
%%%     width/.store in=\cubex,
%%%     height/.store in=\cubey,
%%%     depth/.store in=\cubez,
%%%     labelx/.store in=\cubelabelx,
%%%     labely/.store in=\cubelabely,  
%%%     labelz/.store in=\cubelabelz,
%%%     scale/.store in=\cubescale,
%%%     width=10,
%%%     height=10,
%%%     depth=10,
%%%     labelx=x,
%%%     labely=y,
%%%     labelz=z,
%%%     scale=.1,
%%%   }
